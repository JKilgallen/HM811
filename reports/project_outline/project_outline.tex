\documentclass[english]{article}
\usepackage[T1]{fontenc}
\usepackage[latin9]{inputenc}
\usepackage{babel}
\begin{document}
\title{HM811 - Computational Tools for Research}
\author{J. A. Kilgallen}
\maketitle

\section{Introduction}

The HM811 assignment is to submit 3 programs which carry out tasks
useful to my research, each of which must be in a different programming
language. The available languages to choose from are:
\begin{itemize}
\item C
\item Bash
\item Perl
\item \LaTeX
\item Octave/MATLAB
\item R
\end{itemize}

\section{Project Structure }

The structure I've decided for my project is to build a docker image
which will:
\begin{itemize}
\item Install the required packages and libraries.
\item Pull Octave code from a repository. 
\item Run an Octave script to generate several plots as PDFs, and a series
of images as PNGs.
\item Render the PNGs into an animation using the FFMPEG library.
\item Compile a \LaTeX document which includes the animation, and the plots
generated in the previous step.
\end{itemize}

\section{Project Content}

The content of the document which is the final output will be a brief
overview of a class of optical illusions known as MacKay effects (including
proper bibtex referencing of significant papers), which includes high-quality
graphics of static stimuli which exhibit the effect. The second portion
of the document will discuss future work on the MacKay effect and
will include dyanimic stimuli in the form of embedded MPEGs.

\section{Project Stages}

The project will have three primary stages:
\begin{enumerate}
\item Building a basic docker image.
\item Writing Octave code to generate relevant plots.
\item Writing \LaTeX document.
\end{enumerate}

\end{document}
